%\documentclass[preprint]{aastex}
\documentclass{emulateapj}
\usepackage{amsmath}
\usepackage{graphicx}
\usepackage{amsfonts}
\usepackage{natbib}

\bibliographystyle{apj}

%\textwidth 6.5in
%\textheight 8.5in
%\topmargin = 0mm
%\evensidemargin = 0mm
%\oddsidemargin = 0mm

%Custom Commands
\newcommand{\al}{\alpha}
\newcommand{\be}{\beta}
\newcommand{\gam}{\gamma}
\newcommand{\Gam}{\Gamma}
\newcommand{\de}{\delta}
\newcommand{\De}{\Delta}
\newcommand{\eps}{\epsilon}
\newcommand{\sig}{\sigma}
\newcommand{\Sig}{\Sigma}
\newcommand{\ka}{\kappa}
\newcommand{\lam}{\lambda}
\newcommand{\pd}{\partial}
\newcommand{\dd}{\mbox{d}}

\newcommand{\scalefit}{{\texttt{ScaleFit}}}
\newcommand{\boxfit}{{\texttt{BoxFit}}}
\newcommand{\ramcode}{{\texttt{RAM}}}
\newcommand{\blastcode}{{\texttt{BLAST}}}
\newcommand{\DISCO}{{\texttt{DISCO}}}
\newcommand{\swift}{{\it Swift}}
\newcommand{\swiftXRT}{{\it Swift-XRT}}
\newcommand{\XRT}{{\it XRT}}
\newcommand{\vela}{{\it Vela}}
\newcommand{\bepposax}{{\it BeppoSAX}}
\newcommand{\chandra}{{\it Chandra}}
\newcommand{\ram}{{\texttt{RAM}}}

\newcommand{\OO}{\mathcal{O}}

\begin{document}

\title{Minidiscs in Circumbinary Black Hole Accretion}
\author{Geoffrey Ryan\altaffilmark{1,a} and Andrew MacFadyen\altaffilmark{1}}
\altaffiltext{a}{gsr257@nyu.edu}
\altaffiltext{1}{Center for Cosmology and Particle Physics, Physics Department, New York University, NY, NY 10003, USA}

\begin{abstract}

ABSTRACT ABSTRACT ABSTRACT

\end{abstract}


%%%%%
%Section 1 - Introduction
%%%%%

\section{Introduction}
\label{sec:intro}

Binary black holes are awesome!  

Since galaxies form hierarchically there should be a lot of supermassive ones maybe!  

But, you know, final parsec problem?

Global circumbinary accretion still an active problem, uncertain when it aids
or hinders black hole mergers.

Observationally, need EM counterparts to identify SMBBHs.  EM counterparts 
probably accretion powered.

Also, you know, LIGO.  GW150914 and others.  Stellar mass BBHs, EM counterparts
will almost certainly involve some flavour of circumbinary accretion.

Newtonian work has been done on global discs.  Seems in indicate enhanced
accretion rates and negative(?) torques on the binary.  Broad structure is
quasi-stationary Shakura-Sunyaev disc in outer regions. Possibly eccentric
cavity near BHs. Ballistic streams from cavity edge feed minidiscs around each BH, with a bridge flowing between them.

Newtonian simulations are necessarily underresolved near each BH, approximate accretion prescriptions are introduced to prevent artifical accumulation of mass.

In this work we present the results of two dimensional inviscid general relativistic
hydrodynamic simulations of accretion discs around a black hole in a binary.
They are built to model minidics seen in global circumbinary calculations,
and can be seen as a ``zoomed in'' calculation of the hydrodynamics in the 
immediate vicinity of one of the black holes.

Newtonian calculations have been performed with alpha viscosity and either isothermal \citep{Farris14} or adiabatic \citep{Farris15A, Farris15B} thermodynamics. The adiabatic calculations included local blackbody cooling to dissipate the
viscous heat generation, and were shown to lead to the Shakura-Sunyaev disc
when performed with a single central black hole.

Although alpha viscosity is typically used to model the unknown angular 
momentum transport mechanism in accretion discs, we find minidiscs accrete with
ideal hydrodynamics alone.  This is due to the presence of spiral waves (shocks?) excited by the binary companion. These shocks transport angular momentum outwards (cite Binney \& Tremaine?) and locally heat the disc. We include local 
black-body cooling with electron-scattering opacity to remove shock-generated
heat self-consistently from the disc.  This allows the disc to find a natural
temperature equilibrium, and allows a direct estimate of the emission spectrum.

This paper is organized as follows in Section \ref{sec:numerics} we present the
numerical setup used in the calculations; a version of the \DISCO{} code 
modified to work in an arbitrary space-time, with optimizations for thin 
relativistic accretion discs.  In Section \ref{sec:models} we detail the 
minidisc models calculated.  Section \ref{sec:results} analyzes the models and
calculates effective $\alpha$'s and spectra. Results are discussed in Section \ref{sec:discussion} and the work is summarized in \ref{sec:summary}.

%%%%%
%Section 2 - Numerical Setup
%%%%%

\section{Numerical Setup}
\label{sec:numerics}

The basis of our hydrodynamics scheme is the \DISCO{} code; a moving mesh hydro
code optimized for disc geometry. This code was first used in the context of
protoplanetary discs \citep{Duffell12, Duffell13, Duffell14} and later applied 
to circumbinary accretion \citep{Farris14, Farris15A, Farris15B}. 

The original \DISCO{} code has been modified to solve the GRHD equations in a 
fixed space-time:
\begin{equation}
    \nabla_\mu \rho_0 u^\mu = 0 \text{ and } \nabla_\mu T^{\mu\nu} = -\dot{Q} u^\nu , \label{eq:GRHD}
\end{equation}
for a single species gas of rest-mass density $\rho_0$, four velocity $u^\mu$, 
stress energy tensory $T^{\mu\nu}$ and local isotropic cooling $\dot{Q}$.  

To solve \eqref{eq:GRHD} numerically one must make a choice of which elements of $T^{\mu\nu}$ to be independent variables.  We follow the standard Valencia formulation (CITE) (also cite HARM, Duez2004, etc) for the momentum variables $T^0_i$ and choose an energy variable projected onto a known four-velocity $U^\mu$: $-U_\mu T^{\mu 0}$.
In terms of coordinate derivatives \eqref{eq:GRHD} takes the standard flux-balanced conservation form
\begin{equation}
    \pd_0 \mathcal{U} + \pd_j \mathcal{F}^j = \mathcal{S} ,
\end{equation}
with conserved variables
\begin{equation}
    \mathcal{U} = \begin{pmatrix} D \\
                            S_i \\
                            \tau_U
                \end{pmatrix} = \sqrt{-g} \begin{pmatrix} \rho_0 u^0 \\ 
                                                    T^0_i \\
                                                    -U^\mu T_\mu^0 - \rho_0 u^0 \end{pmatrix} , \label{eq:cons}
\end{equation}
fluxes
\begin{equation}
    \mathcal{F}^j = \sqrt{-g} \begin{pmatrix} \rho_0 u^j \\
                                                T^j_i \\
                                                -U^\mu T_\mu^j \end{pmatrix} ,\label{eq:fluxes}
\end{equation}
and source terms 
\begin{equation}
    \mathcal{S} = \sqrt{-g} \begin{pmatrix} 0 \\
                        \frac{1}{2}T^{\mu\nu}\pd_i g_{\mu\nu} - \dot{Q}u_i \\
                        T^{\mu\nu}\nabla_\mu U_\nu + U^\mu u_\mu \dot{Q} \end{pmatrix} .\label{eq:sources}
\end{equation}

In this work we assume an ideal gas with stress tensor
\begin{equation}
	T^{\mu\nu} = \rho_0 h u^\mu u^\nu + P g^{\mu\nu} ,
\end{equation}
where $P$ is the gas pressure, $h = 1 + \eps + P/\rho_0$ the relativistic specific enthalpy, and $\eps$ the specific internal energy. Furthermore we assume the gamma law equation of state
\begin{equation}
	P = (\Gam - 1) \rho_0 \eps .
\end{equation}

\subsection{Choice of Energy Variable}
\label{sec:energy}







%%%%%
%Section 3 - Minidisc Models
%%%%%

\section{Minidisc Models}
\label{sec:models}



%%%%%
%Section 4 - Results
%%%%%

\section{Results}
\label{sec:results}



%%%%%%
% Discussion %
%%%%%%

\section{Discussion}
\label{sec:discussion}



%%%%%%
% Summary %
%%%%%%

\section{Summary}
\label{sec:summary}




%%%%%%
% Acknowledgements %
%%%%%%

\section{Acknowledgements}


\newpage

%%%%%%
% Bibliography %
%%%%%%

\bibliography{minidisc_sources}

%%%%%%
% Appendix %
%%%%%%

%\include{appendix}

\end{document}
