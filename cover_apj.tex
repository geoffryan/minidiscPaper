\documentclass{letter}
\usepackage{setspace}

\begin{document}

\begin{flushleft}
\begin{singlespace}
Department of Physics\\
New York University\\
4 Washington Place, Room 424\\
New York, NY 10003
\end{singlespace}

\vspace{0.5cm}

Nov 1, 2016

\vspace{0.5cm}

Dear Editor,

%Please find enclosed our paper ``Minidisks in Binary Black Hole Accretion,'' which we are submitting exclusively to the Astrophysical Journal.  Please contact me for any correspondence concerning the manuscript, my email address is \texttt{gsr257@nyu.edu}.

Please find enclosed our revised paper ``Minidisks in Binary Black Hole Accretion.'' The changes suggested by the referee report have been implemented, and references to Artymowicz \& Lubow 1994 as well as Zhu 2016 have been added.

Following are the referee's comments and our responses.

Introduction
\begin{enumerate}
\item In the third line there is a typo. Where it says: ``and are expected to be be'' there is an additional ``be''.
	\begin{itemize}
		\item The typo has been removed. 
	\end{itemize}

\item In the third paragraph where the formation of a cavity is discussed it would be worth mentioning that the cavity will be formed only if certain conditions are fulfilled and that in some cases we may expect a binary to be totally embedded in a gaseous environment without the existence of a cavity. For example you can see this in common envelope systems (Passy etal. 2011, Ricker and Taam 2012) or in SMBHB (Dotti 2006). You can also read about the conditions for gap formation in del Valle \& Escala 2014.
	\begin{itemize}
		\item The first sentence of paragraph 3 of the introduction has been modified to qualify that cavities in circumbinary disks are only formed under particular circumstances.  A paragraph has been added to the end of the introduction indicating the focus of this work is on circumbinary accretion where a cavity has formed, including a reference to del Valle \& Escala 2014.
	\end{itemize}
\end{enumerate}

Numerical setup
\begin{enumerate}
\item At the end of the page 2 and at the beginning of page 3 where it says HLLC and CFL it would be better to define the meaning of the initials (for example Courant Factor Limit for CFL).
	\begin{itemize}
		\item The acronyms have been defined where they first appear.
	\end{itemize}
\end{enumerate}

Minidisk Models
\begin{enumerate}
\item At the end of page 5 when the authors discuss the accretion rate for a system with M=Msun it will be useful to have also the information of the binary separation for this mass. Indeed, it would be important to present, in an explicit way, the separation of the binary for masses of SMBHBs like the ones that we expect to observe with eLISA (Mbin ~ 1e6 Msun), or BHBs as the ones observed by LIGO. Also, the authors may say something about the evolution of the binary separation due to gravitational wave emission. For example, if I'm not mistaken, for a SMBHB with M=1e6 Msun the binary separation that the authors define will be a=1000*M=1000*1e6*G/(c*c)=4.7e-5 pc. For this distance we expect that the binary is emitting gravitational waves and will coalescence in something like 4e4 orbits (tgw=124 yr), therefore in the 20-30 orbits in which the authors evolve the system the binary separation can be assumed as constant, as they do (the authors can estimate this using the expressions in Peters 1964 for da/dt). This estimation would clarify the validity of the assumption of the authors.
	\begin{itemize}
		\item A paragraph has been added to section 3 giving the formula for merger time due to gravitational wave emission (from Peters 1964) and justifying our neglect of evolution of the binary orbital parameters. Two sentences giving the merger and orbital time for the $M=1 M_{\odot}$  and $M=10^6 M_{\odot}$ cases have been added to the last paragraph of section 3.
	\end{itemize}
\end{enumerate}

Discussion
\begin{enumerate}
\item In the discussion I think the authors may add some lines about two main points: i) Some words about the evolution of the binary separation and if this can change their results (see previous point of minidisk models). ii) How, or if, the structure of the minidisk may change if there is no cavity in the circumbinary disk (in this case the accretion onto the minidisk may no be restricted to one stream of gas).
	\begin{itemize}
		\item A paragraph has been added to the end of section 6, discussing our results in the context of evolution of the binary parameters.  Circumbinary accretion with no cavity is a very different problem and this work cannot reliably be used to say much about it, as we now make clear in the introduction.		
	\end{itemize}
\end{enumerate}

  Please contact me for any correspondence concerning the manuscript, my email address is \texttt{gsr257@nyu.edu}.

\vspace{0.5cm}

Sincerely,

Geoffrey Ryan
\end{flushleft}

\end{document}
